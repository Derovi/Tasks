\documentclass[a4paper,12pt]{article}

\usepackage[left=2cm,right=2cm,
top=2cm,bottom=2cm,bindingoffset=0cm]{geometry}
\usepackage{cmap}					% поиск в PDF
\usepackage[T2A]{fontenc}			% кодировка
\usepackage[utf8]{inputenc}			% кодировка исходного текста

\usepackage{xcolor}
\usepackage{titlesec}
\titleformat{\section}[block]{\Large\bfseries\filcenter}{}{1em}{}
\titleformat{\subsection}[block]{\color{blue}\Medium\bfseries\filcenter}{}{1em}{}
\usepackage{natbib}
\usepackage{esvect}
\usepackage{graphicx}
\usepackage{amsmath}
\usepackage{dsfont}
\usepackage[russian]{babel}

\begin{document}

\section{Problem 66}

Из условия ясно, что граф является DAG'ом (ориентированным графом без циклов) - \textit{"Из-за карантина на дорогах введено одностороннее движение. Вася  — опытный дезинфектор, и поэтому знает, что выехав из любого пункта и проехав несколько дорог, невозможно приехать опять в тот же пункт."}. 

Отсюда можно сделать вывод, что если есть какой-то подходящий под условие путь от дез-станции до дез-здрав пункта, который проходит через вершину $t$, при этом до вершины $t$ этот путь проходит через какое-то подмножество домов множества $c_1,c_2,...,c_n$, то любой другой путь тоже будет проходить через эти дома. Доказать это легко. Если до посещения вершины $t$ мы могли посетить некоторые дома, но этого не сделали, мы не сможем вернуться и посетить их после вершины $t$, так как граф не имеет циклов. 

Будем проходится в порядке топологичнской сортировки и считать количество способов добраться до каждой вершины, посетив все дома из $c_1,c_2,...,c_n$, которые можно было посетить на пути к этой вершине. Ответ будет лежать в конечной вершине.


\end{document}
