\documentclass[a4paper,12pt]{article}

\usepackage[left=2cm,right=2cm,
top=2cm,bottom=2cm,bindingoffset=0cm]{geometry}
\usepackage{cmap}					% поиск в PDF
\usepackage[T2A]{fontenc}			% кодировка
\usepackage[utf8]{inputenc}			% кодировка исходного текста

\usepackage{xcolor}
\usepackage{titlesec}
\titleformat{\section}[block]{\Large\bfseries\filcenter}{}{1em}{}
\titleformat{\subsection}[block]{\color{blue}\Medium\bfseries\filcenter}{}{1em}{}
\usepackage{natbib}
\usepackage{esvect}
\usepackage{graphicx}
\usepackage{amsmath}
\usepackage{dsfont}
\usepackage[russian]{babel}

\begin{document}

\section{Problem 63}

Из условия, нам необходимо выбрать подооигртрезок игр $[a, b]$ максимальной длины такой, что для всех $t \in [a, b]$ верно, что $t_r >= e_l$, где $e \in [a, t)$, т.e. в любом отрезке читеров правая граница больше, чем максимум из левых границ в предыдущих отрезках.

Таким образом, решение за квадрат выглядит так - перебираем $b$ - правую границу отрезка игр, затем перебираем $a$ от $b$ до 0, пока $a_r >= b_l$. Как только это условие не выполняется, найдет отрезок игр максимальной длины с правой границей в $b$. 

Это решение можно улучшить, используя очередь на максимум. Будем перебирать $t$ - номер игры, добавлять в чередь $t_l$ и удалять из очереди элементы, пока максимум в очереди на станет $<= t_r$. Количество элементов в очереди после такой операции - максимальная длина отрезка игр с правой границей в $t$.

Сложность такого решения $O(N)$, так как всего в очередь будет добавлено $N$ элементов, значит удалено не более $N$. В очереди на максимум операции добавления элемента, удаления элемента, поиска максимума работают за $O(1)$.

\end{document}
