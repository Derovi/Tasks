\documentclass[a4paper,12pt]{article}

\usepackage[left=2cm,right=2cm,
top=2cm,bottom=2cm,bindingoffset=0cm]{geometry}
\usepackage{cmap}					% поиск в PDF
\usepackage[T2A]{fontenc}			% кодировка
\usepackage[utf8]{inputenc}			% кодировка исходного текста

\usepackage{xcolor}
\usepackage{titlesec}
\titleformat{\section}[block]{\Large\bfseries\filcenter}{}{1em}{}
\titleformat{\subsection}[block]{\color{blue}\Medium\bfseries\filcenter}{}{1em}{}
\usepackage{natbib}
\usepackage{esvect}
\usepackage{graphicx}
\usepackage{amsmath}
\usepackage{dsfont}
\usepackage[russian]{babel}

\begin{document}

\section{Problem 34}

В задаче на каждую вершину накладывается два условия:

1) Расстояние до любой другой вершины меньше k.

2) Нет полупути, который имеет длину $k$ и проходит через эту вершину.

Второе условие покрывает первое, т.e. если для вершины $a$ есть вершина $b$ такая, что расстояние от $a$ до $b$ равно $k$, то существует полупуть длины хотя бы $k$ (который проходит через $a, b$).

Таким образом, для каждой вершины надо проверить есть ли полупуть, который проходит через нее и имеет длину $>=k$, а затем из всех подходящих вершин найти максимальную и удалить ее правым удалением.

Для каждой вершины найдем $height$ - максимальное расстояние до вершин в поддеревьях.

Будем для каждой вершины $t$ перебирать предка $p$ и считать, что полупуть проходит через $p$ и не проходит через предков вершины $p$, т.e. идет от самого дальнего из потомков вершины $t$ до вершины $t$, затем от вершины $t$ до вершины $p$, а затем от вершины $p$ до самого дальнего ее потомка в поддереве, в котором нет вершины $t$.

Сложность работы алгоритма $O(n^2)$, так как для кажой вершины нужно произвести количество операций, равное расстоянию от этой вершины до корня дерева. Если дерево в условие бы было сбалансированным, то сложность работы алгоритма была бы $O(n*log_2(n))$

\end{document}
