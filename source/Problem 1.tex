\documentclass[a4paper,12pt]{article}

\usepackage[left=2cm,right=2cm,
top=2cm,bottom=2cm,bindingoffset=0cm]{geometry}
\usepackage{cmap}					% поиск в PDF
\usepackage[T2A]{fontenc}			% кодировка
\usepackage[utf8]{inputenc}			% кодировка исходного текста

\usepackage{xcolor}
\usepackage{titlesec}
\titleformat{\section}[block]{\Large\bfseries\filcenter}{}{1em}{}
\titleformat{\subsection}[block]{\color{blue}\Medium\bfseries\filcenter}{}{1em}{}
\usepackage{natbib}
\usepackage{esvect}
\usepackage{graphicx}
\usepackage{amsmath}
\usepackage{dsfont}
\usepackage[russian]{babel}

\begin{document}

\section{Problem 1}


Будем считать, что у нас есть события двух видов:

1) Деталь с номером $i$ поступила

2)  Станок с номером $j$ освободился

Будем поддерживать список свободных станков и деталей, которые уже поступили, но еще не отправлены на станок.

После наступления каждого события будем брать деталь с максимальным временем доставки на склад и отправлять на любой из свободных станков, пока есть не распределенные детали или свободные станки.

Таким образом, мы исключаем ситуацию, когда деталь, которая поступила подздно и имеет большое время доставки на склад, будет долго 
ждать станка. Однако, может быть ситуация, в которой было бы разумнее задержать какую-то деталь из пришедших ранее, чтобы раньше отправить на станок деталь с большим временем доставки на склад, так что ответ может отличаться от оптимального, но не более, чем в два раза.

\end{document}
